\documentclass[a4paper, 11pt]{scrartcl}

\usepackage{ngerman}

\usepackage[utf8]{inputenc}

\usepackage[T1]{fontenc}

\usepackage{ae,aecompl}

\usepackage{amsmath,amssymb,amstext}

\usepackage{psfrag}

\usepackage{biblatex}

\bibliography{library}

\usepackage[automark]{scrpage2}

\usepackage{ifpdf}

\ifpdf%   (definitions for using pdflatex instead of latex)

  %%% graphicx: support for graphics
  \usepackage[pdftex]{graphicx}

  \pdfcompresslevel=9

  %%% hyperref (hyperlinks in PDF): for more options or more detailed
  %%%          explanations, see the documentation of the hyperref-package
  \usepackage[%
    %%% general options
    pdftex=true,      %% sets up hyperref for use with the pdftex program
    %plainpages=false, %% set it to false, if pdflatex complains: ``destination with same identifier already exists''
    %
    %%% extension options
    backref,      %% adds a backlink text to the end of each item in the bibliography
    pagebackref=false, %% if true, creates backward references as a list of page numbers in the bibliography
    colorlinks=true,   %% turn on colored links (true is better for on-screen reading, false is better for printout versions)
    %
    %%% PDF-specific display options
    bookmarks=true,          %% if true, generate PDF bookmarks (requires two passes of pdflatex)
    bookmarksopen=false,     %% if true, show all PDF bookmarks expanded
    bookmarksnumbered=false, %% if true, add the section numbers to the bookmarks
    %pdfstartpage={1},        %% determines, on which page the PDF file is opened
    pdfpagemode=None         %% None, UseOutlines (=show bookmarks), UseThumbs (show thumbnails), FullScreen
  ]{hyperref}


  %%% provide all graphics (also) in this format, so you don't have
  %%% to add the file extensions to the \includegraphics-command
  %%% and/or you don't have to distinguish between generating
  %%% dvi/ps (through latex) and pdf (through pdflatex)
  \DeclareGraphicsExtensions{.pdf}

\else %else   (definitions for using latex instead of pdflatex)

  \usepackage[dvips]{graphicx}

  \DeclareGraphicsExtensions{.eps}

  \usepackage[%
    dvips,           %% sets up hyperref for use with the dvips driver
    colorlinks=false %% better for printout version; almost every hyperref-extension is eliminated by using dvips
  ]{hyperref}

\fi


%%% sets the PDF-Information options
\hypersetup{
  pdftitle={},
  pdfauthor={}, %%
  pdfsubject={}, %%
  pdfcreator={Accomplished with LaTeX2e and pdfLaTeX with hyperref-package.}, %%
  pdfproducer={}, %%
  pdfkeywords={} %%
}

%%%%%%%%%%%%%%%%%%%%%%%%%%%%%%%%%%%%%%%%%%%%%%%%%%%%%%%%%%%%%%%%%%%%%%%%%%%%%%%%
%%%
%%% define the titlepage
%%%

% \subject{}   %% subject which appears above titlehead
% \titlehead{} %% special heading for the titlepage

%%% title
\title{Intelligente Systeme Ausarbeitung}

%%% author(s)
\author{Florian Nehmer (Matr.Nr.: 2193399)}

%%% date
\date{\today}


%%%%%%%%%%%%%%%%%%%%%%%%%%%%%%%%%%%%%%%%%%%%%%%%%%%%%%%%%%%%%%%%%%%%%%%%%%%%%%%%
%%%
%%% set heading and footer
%%%

%%% scrheadings default:
%%%      footer - middle: page number
\pagestyle{scrheadings}


%%%%%%%%%%%%%%%%%%%%%%%%%%%%%%%%%%%%%%%%%%%%%%%%%%%%%%%%%%%%%%%%%%%%%%%%%%%%%%%%
%%%
%%% begin document
%%%

\begin{document}

%%% include the title
\maketitle

\newpage
\tableofcontents

\newpage

%%%%%%%%%%%%%%%%%%%%%%%%%%%%%%%%%%%%%%%%%%%%%%%%%%%%%%%%%%%%%%%%%%%%%%%%%%%%%%%%
%%%
%%% begin main document
%%% structure: \section \subsection \subsubsection \paragraph \subparagraph
%%%

\section{Aufgabe Suchen, Lernen, NLP; Jeweils $\frac{1}{2}$ Seite}
noch ein Hinweis zu Ihren Ausarbeitungen: Für 'Suchen', 'Lernen' und 'Verarbeitung natürlicher Sprache' soll nur eine Stichwortliste
abgegeben werden, bitte kein Fließtext. Es geht darum, dass Sie sich selbst klar machen, welche Konzepte Sie sich erarbeitet haben.
Ein Beispiel wie so etwas aussehen kann für 'Suchen'

\begin{itemize}
\item A*-Algorithmus im Detail
\item IDA*-Algorithmus im Detail
\item Zulässigkeit
\item Monotonie
\item Heuristiken
\item Eigenschaften der beiden Algorithmen: Vollständigkeit, Optimalität
\end{itemize}

\subsection{Suchen}
\begin{itemize}
  \item Monte-Carlo-Tree-Search im Detail
  \item UCB1-Formel
  \item Exploration vs. Exploitation
  \item Eigenschaften des Algorithmus: OPtimalität
\end{itemize}

\subsection{Lernen}
\begin{itemize}
  \item Deep Q-Learning
  \item Markov Entscheidungsprozess
  \item $\epsilon$-Greedy
  \item Q-Value
  \item Q* approximieren mithilfe eines neuronalen Netz
\end{itemize}

\subsection{NLP}
\begin{itemize}
  \item ?
\end{itemize}

\newpage

\section{Was macht Intelligenz aus?}

2014 prophezeite Steven Hawking der BBC, dass die Entwicklung voller künstlicher Intelligenz,
das Ende der Menschheit bedeuten wird, als Teil seiner Antwort auf eine Frage über sein Kommunikationsgerät, welches
mit Hilfe eines KI Systems funktionierte.

Sind diese heute schon realen KI Systeme die Vorläufer einer vollen künstlichen Intelligenz? Ist es überhaupt möglich,
dass Maschinen mit einer vollen KI an die menschliche Intelligenz heran kommen? Was macht eigentlich Intelligenz aus?
Diesen Fragen soll nachfolgender Text auf den Grund gehen.
Zuerst werden Definition des Begriffes Intelligenz beleuchtet. Dafür werden verschiedene Intelligenzmodelle aus dem
Bereich der Psychologie herangezogen, um letztendlich ein Bild über den Begriff Intelligenz zu erhalten.
Desweiteren wird der Begriff emotionale Intelligenz skizziert und mit dem Begriff Intelligenz unter menschlicher
Intelligenz zusammengefasst.

Nachfolgend beschäftigt sich der Text damit, was heutzutage unter dem Begriff küstlicher Intelligenz verstanden wird.
Im Anschluss wird die menschliche Intelligenz, der künstlichen Intelligenz gegenüber gestellt, um zu erörtern, was
menschliche Intelligenz ausmacht. Zum Schluss wird noch ein Gedankenspiel skizziert, was mit Maschinen theoretisch
möglich sein könnte.

\subsection{Intelligenz}
Der Begriff Intelligenz hat keine algemeingültige Definition. In der Psychologie sind im Laufe der Zeit viel mehr verschiedene
Intelligenzmodelle entstanden. Im Folgenden werden drei Intelligenzmodelle herangezogen, um verschiedene Facetten des
 Intelligenzbegriffes kennenzulernen

\subsubsection{Zwei-Faktoren-Modell von Horn}
 Zu den vorherrschenden Theorien gehört zum Beispiel das Zwei-Faktoren-Modell von Horn und Cattle.
Dieses beschreibt 2 Arten von Intelligenz. Die Kristalline Intelligenz, welche die Erfahrungen darstellt, die ein Mensch
im Laufe seines Lebens sammelt und die Fakten, die er dadurch lernt. Die kristalline Intelligenz sei stark kulturell beeinflusst.
Die zweite der beiden Arten ist die fluide Intelligenz, welche
die Fähigkeit eines Menschens repräsentiert, sich Fakten und Erfahrungen anzueigenen, also das Denken. Ein hohes Maß
an fluider Intelligenz sei notwendig, um sich schnell in unbekannten Situationen zurecht zu finden. Die fluide Intelligenz sei genetisch
determiniert. Die fluide Intelligenz nehme ab dem 25. Lebensjahr ab und die kristalline Intelligenz steige bis zum 25. Lebenjahr stark an,
jedoch danach nur noch langsam. (vgl. \cite{Dorsch2019}). Es zeigen sich in diesem Modell schon zwei sehr grobe Facetten
des Begriffes Intelligenz.

\subsubsection{Sternbergs triarchisches Modell}
Ein weiteres wichtiges Intelligenzmodell lieferte
der Psychologe Robert J. Sternberg, nämlich das sogenannte triarchische Modell auch unter dem Namen Komponentenmodell
bekannt. Sternberg sieht einen Zusammenhang zwischen Intelligenz und Erfolg im Leben und unterteilt Intelligenz in 3 Bereiche:
Analytische-, Praktische- und Erfahrungsbezogene Fähigkeiten. Er sieht also die Intelligenz mehr als einen Prozess, also die Art und
Weise wie Informationen verarbeitet werden. Im Gegensatz zum Modell zuvor werden hier alleine die Fähigkeiten in den Vordergrund gestellt.
Wobei die Erfahrungen als Fähigkeit betrachtet werden und incht als Schatz, wie in der Theorie von Horn und Cattle. (vgl.\cite{Stern1984})

\subsubsection{Multiple Intelligenzen nach Gardner}
Als letztes Modell der menschlichen Intelligenz betrachten wir das Modell der multiplen Intelligenzen von Gardner. Dieser formulierte
ein noch feingranulareres Modell der Intelligenz und kategorisiert den Intelligenzbegriff in 8 Unterkategorien.
(vgl. \cite{Gardner1993})




Wie kann man nun diese Modell auf Computer übertragen?
Also auf der Hand liegt, dass Computer sehr wohl in der Lage sind ihren Wissenschatz kontinuierlich zu erweitern
Einzig und allein begrenzt durch Speicherkapazität. Computer können bezogen auf Fakten einen großen
Wissensschatz aufbauen. Auch Erfahrungen können zumindet abgespeichert werden. Jedoch Schlüsse aus diesen
Erfahrungen zu ziehen, um diese in intelligent in neuen Situationen einzusetzen

\newpage



\section{Was kann ich tun, um als Informatiker verantwortlich zu handeln?; $\frac{1}{2}$ Seite}

\newpage

\printbibliography












%%%
%%% end main document
%%%
%%%%%%%%%%%%%%%%%%%%%%%%%%%%%%%%%%%%%%%%%%%%%%%%%%%%%%%%%%%%%%%%%%%%%%%%%%%%%%%%

% \appendix  %% include it, if something (bibliography, index, ...) follows below

%%%%%%%%%%%%%%%%%%%%%%%%%%%%%%%%%%%%%%%%%%%%%%%%%%%%%%%%%%%%%%%%%%%%%%%%%%%%%%%%
%%%
%%% bibliography
%%%
%%% available styles: abbrv, acm, alpha, apalike, ieeetr, plain, siam, unsrt
%%%
% \bibliographystyle{plain}

%%% name of the bibliography file without .bib
%%% e.g.: literatur.bib -> \bibliography{literatur}
% \bibliography{FIXXME}

\end{document}
